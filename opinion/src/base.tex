\documentclass[12pt]{article}

\usepackage[T2A]{fontenc}
\usepackage[utf8]{inputenc}
\usepackage[english,russian]{babel}
\usepackage{indentfirst}

\usepackage[babel,final,protrusion=true,expansion]{microtype}

\usepackage{amssymb}

\usepackage{../../common/cypokcommon}

\usepackage[a4paper,left=3cm,right=3cm,top=3cm,bottom=3cm,bindingoffset=0cm]{geometry}
\linespread{1.3}

\newcommand\he{Парфиненко\,В.\,В.}

\titleauthor{%
  Аннотация к магистерской диссертации на тему
  «Применение анализа указателей и синонимов
  для оптимизации многопоточных программ»%
}{%
  Владимир Парфиненко%
}

\begin{document}

  \thispagestyle{empty}

  \begin{center}
    \bfseries
    {\Large \scshape Отзыв научного руководителя}\\
    о магистерской диссертации\\
    Парфиненко Владимира Владимировича\\
    {\large
      <<Применение анализа указателей и синонимов\\
      для оптимизации многопоточных программ>>\\
    }
  \end{center}

  Анализ указателей является важным видом статического анализа программ и
  используется при проведении большого числа оптимизаций.
  Задача \he{} заключалась в разработке внутрипроцедурного алгоритма анализа
  указателей для языка \eng{Java}, последующей его реализации и использования
  для проведения других оптимизаций в рамках проекта \eng{Excelsior Research
  Virtual Machine}.

  Для решения поставленной задачи \he{} исследовал существующие алгоритмы
  анализа указателей, спецификация языка \eng{Java} и его модели памяти.
  Затем был разработан способ выражения неявных зависимостей по памяти с
  помощью введения искусственной переменной, представляющей образ всей памяти.
  Результатом этой работы стало создание эффективного алгоритма анализа и
  внутреннего представления, учитывающих особенности, возникающие при анализе
  многопоточных \eng{Java}-программ.

  \he{} выполнил реализацию предложенного алгоритма в рамках оптимизирующего
  статического компилятора.
  Также была разработана и реализована оптимизация удаления чтений полей
  объектов, которая использует результаты анализа указателей.
  Результаты работы данной оптимизации, использующей разные варианты алгоритма
  анализа, позволяют утверждать, что алгоритм, разработанный \he{}, является
  достаточно эффективным для использования в оптимизирующем статическом
  компиляторе.

  В процессе работы магистрант проявил отличные знания теории компиляции,
  навыки системного программиста и продемонстрировал личные качества, такие как
  целеустремленность и хорошую работоспособность.

  Работа заслуживает оценки <<отлично>> и рекомендации на конкурс, а студент
  присуждения степени магистра.
  Также \he{} рекомендуется для дальнейшего обучения в аспирантуре.

  \begin{flushright}
    м.\,н.\,с.~ИСИ~СО~РАН\\
    Павлов\,П.\,Е.
  \end{flushright}

\end{document}
