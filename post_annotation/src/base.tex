\documentclass[12pt]{article}

\usepackage[T2A]{fontenc}
\usepackage[utf8]{inputenc}
\usepackage[english,russian]{babel}
\usepackage{indentfirst}

\usepackage[babel,final,protrusion=true,expansion]{microtype}

\usepackage{amssymb}

\usepackage{../../common/cypokcommon}

\usepackage[a4paper,left=3cm,right=3cm,top=3cm,bottom=3cm,bindingoffset=0cm]{geometry}
\linespread{1.3}

\titleauthor{%
  Аннотация к магистерской диссертации на тему
  «Применение анализа указателей и синонимов
  для оптимизации многопоточных программ»%
}{%
  Владимир Парфиненко%
}

\begin{document}

  \thispagestyle{empty}

  \begin{center}
    \bfseries
    {\Large \scshape Аннотация}\\
    к магистерской диссертации\\
    Парфиненко Владимира Владимировича\\
    {\large
      <<Применение анализа указателей и синонимов\\
      для оптимизации многопоточных программ>>\\
    }
  \end{center}

  Анализ указателей и синонимов является одним из видов статического анализа.
  Результаты такого анализа могут быть использованы для усиления других видов
  анализа, проводимых оптимизирующим компилятором.
  Таким образом может быть усилен достаточно обширный класс анализов и
  оптимизаций, как классических, так и объектно-ориентированных.

  В программах на языке \eng{Java} семантика любого участка кода программы
  зависит от многопоточного окружения.
  Взаимодействие различных потоков исполнения друг с другом и с разделяемой
  памятью описывается моделью памяти данного языка\slash{}среды, которая
  определяет корректность тех или иных преобразований программы.
  Корректное и при этом не чрезмерно препятствующее оптимизациям определение
  зависимостей и связей в программе возможно только при проведении
  нетривиального анализа указателей и синонимов.

  В данной работе для выражения неявных зависимостей по памяти была
  использована $\mathbb{M}$"=переменная, представляющая образ памяти.
  Семантика всех операций языка \eng{Java}, взаимодействующих с памятью,
  выражена в терминах указателей и $\mathbb{M}$"=переменной, в соответствии с
  моделью памяти языка.
  Анализ указателей сведен к задаче анализа потоков данных: сконструированы
  полурешетки свойств и описаны потоковые функции.

  Дополнительно была разработана оптимизация удаления чтений полей объектов,
  которая использует результаты анализа синонимов.
  Проведение этой оптимизации, в свою очередь, позволяет уточнить результаты
  анализа.

  Таким образом, разработан эффективный алгоритм анализа указателей, пригодный
  для анализа многопоточных программ на языке \eng{Java}.

  \vspace{0.5cm}

  \textbf{Ключевые слова}: статический анализ, оптимизация программ, анализ
  указателей, анализ синонимов, анализ потока данных, многопоточные программы,
  \eng{Java}, \eng{JVM}.

  \textbf{Объем работы}: 36 страниц, 3 рисунка, 3 таблицы, использовано 18
  источников.

  \begin{flushright}
    Парфиненко\,В.\,В.
  \end{flushright}

\end{document}

