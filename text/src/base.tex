\documentclass[14pt,titlepage]{extarticle}
\usepackage[pdftex,unicode,hidelinks]{hyperref}
\usepackage{ucs}
\usepackage[utf8x]{inputenc}
\usepackage[english,russian]{babel}
\usepackage{indentfirst}
\usepackage[usenames,dvipsnames]{color}
\usepackage{amsmath}
\usepackage{amssymb}
\usepackage{mathtools}
\usepackage{multicol}
\usepackage{rotating}
\usepackage{array}
\usepackage{multirow}
\usepackage{xspace}
\usepackage{subfig}
\usepackage{amsthm}

% gnuplot & lua should be installed,
% this *.sty file is generated by `lua %GNUPLOT%/lua/gnuplot-tikz.lua style`
\usepackage{gnuplot-lua-tikz}

\usepackage[left=3cm,right=2cm,top=2cm,bottom=2cm,bindingoffset=0cm]{geometry}
\linespread{1.3}

\usepackage{numprint}
\newcommand{\num}[1]{\numprint{#1}}
  \npthousandsep{\,}
  \npthousandthpartsep{}
  \npdecimalsign{,}

\usepackage{tikz}
\usetikzlibrary{positioning,arrows,shapes}

\usepackage{footmisc}
\renewcommand{\footnotelayout}{\small}

\usepackage{etoolbox}
\apptocmd{\sloppy}{\hbadness 10000\relax}{}{}

% used by bibliography:
\usepackage{datetime}
\newcommand{\usedate}[3]{({\Russian дата обращения: \formatdate{#1}{#2}{#3}})}
\bibliographystyle{gost71u2003} % can be found in the root of repo

%\setcounter{tocdepth}{2} % глубина оглавления

\newcommand{\M}{\ensuremath{\mathbb{M}}}

\newcommand{\NEW}{\textbf{new}}
\newcommand{\NULL}{\textbf{null}}
\newcommand{\INITIALMEMORY}{\textbf{initialmemory}}
\newcommand{\GETFIELD}{\textbf{getfield}}
\newcommand{\PUTFIELD}{\textbf{putfield}}
\newcommand{\GETSTATIC}{\textbf{getstatic}}
\newcommand{\PUTSTATIC}{\textbf{putstatic}}
\newcommand{\ESCAPE}{\textbf{escape}}
\newcommand{\SHARED}{\textbf{shared}}
\newcommand{\RELOAD}{\textbf{reload}}

\newcommand{\Type}[1]{\textrm{Type}(#1)}
\newcommand{\IsAssignable}[2]{\textrm{IsAssignable}(#1, #2)}
\newcommand{\Pts}[1]{\textrm{Pts}(#1)}
\newcommand{\VPts}[1]{\bar{#1}}
\newcommand{\OFPts}[2]{\overline{#1.#2}}
\newcommand{\Shared}{\overline{shared}}
\newcommand{\Filter}[2]{\textrm{Filter}_{#1}(#2)}
\newcommand{\cupe}{\,\cup\!\!=}

\let\oldphi\phi
\renewcommand{\phi}{\ensuremath{\oldphi}}

\renewcommand{\leq}{\leqslant}
\renewcommand{\geq}{\geqslant}
\newcommand{\incomp}{\not\lessgtr}

%\newcommand{\remark}[1]{}
%\newcommand{\todo}[1]{}
%\newcommand{\todocite}{}
\newcommand{\remark}[1]{\textcolor{Green}{(#1)}}
\newcommand{\todo}[1]{\textcolor{red}{(\eng{TODO}: #1)}}
\newcommand{\todocite}{[\textcolor{red}{\eng{cite}}]}

\newcommand{\eng}[1]{{\English#1}}
\newcommand{\engdef}[1]{(англ.~\eng{#1})}

\addto\captionsrussian{
  \let\oldrefname\refname
  \renewcommand\refname{\addcontentsline{toc}{section}{\oldrefname}\oldrefname}
}

% русские нумераторы
\renewcommand{\theenumii}{(\asbuk{enumii})}
\renewcommand{\labelenumii}{\asbuk{enumii})}
\renewcommand{\thesubfigure}{\asbuk{subfigure}}


% обёртка с моими настройками поверх figure:
% \begin{myfigure}{подпись}{label} ... \end{myfigure}
\newenvironment{myfigure}[2]%
  {\pushQED{\caption{#1} \label{#2}} % push caption & label
   \begin{figure}[!htb]\centering } %
  {  \popQED % pop caption & label
   \end{figure}}
\newenvironment{myplot}[2]%
  {\pushQED{\caption{#1} \label{#2}} % push caption & label
   \begin{figure}[p]\centering\small } %
  {  \popQED % pop caption & label
   \end{figure}}

%\newcommand{\inputplot}[1]{\input{#1}}
\newcommand{\inputplot}[1]{Here would be plot}

\let\oldsection\section
\renewcommand{\section}{\newpage\oldsection}

\newcommand{\sectionwithoutnumber}[1]{
  \section*{#1}
  \addcontentsline{toc}{section}{#1}
}

\newcommand{\java}{\eng{Java}\xspace}

\title{
  Применение анализа указателей и синонимов для оптимизации многопоточных программ
}
\author{
  Владимир Парфиненко
}

\begin{document}

  \thispagestyle{empty}
  \begin{center}

    Министерство образования и науки\\
    Российской Федерации

    \vspace{0.7cm}

    Государственное образовательное учреждение\\
    высшего профессионального образования\\
    <<Новосибирский национальный исследовательский\\
    государственный университет>> (НГУ)

    \vspace{0.7cm}

    Механико-математический факультет

    \vspace{0.2cm}

    Кафедра программирования

    \vspace{1.2cm}

    Курсовая работа

    \vspace{0.2cm}

    ПАРФИНЕНКО Владимир Владимирович

    \vspace{1.5cm}

    \textbf{
      ПРИМЕНЕНИЕ АНАЛИЗА УКАЗАТЕЛЕЙ И СИНОНИМОВ\\
      ДЛЯ ОПТИМИЗАЦИИ МНОГОПОТОЧНЫХ ПРОГРАММ
    }

    \vspace{2.5cm}

    \begin{flushright}

      Научные руководители

      м.\,н.\,с.~ИСИ~СО~РАН, Павлов\,П.\,Е.\\
      зав.\,лаб.~ИСИ~СО~РАН, к.\,т.\,н.~Шелехов\,В.\,И.

    \end{flushright}

    \vspace {4cm}

    Новосибирск 2012
  \end{center}

  \tableofcontents

  \sectionwithoutnumber{Операции языка Java}

    При анализе указателей достаточно рассматривать далеко не все операции.
    Тогда мы можем построить вспомогательное внутреннее представление только с
    необходимой и достаточной информацией о потоках данных и зависимостями по
    памяти.

    Приведем список всех рассматриваемых в рамках анализа указателей операций.
    Не зависят от памяти две операции. Создание нового объекта типа
    $\mathrm{T}$:
    \[ x \leftarrow \NEW(\mathrm{T}), \]
    и присваивание специального значения null:
    \[ x \leftarrow \NULL. \]

    Изначальное состояние памяти (\M-переменная) получается специальным
    образом:
    \[ \M_0 \leftarrow \INITIALMEMORY. \]
    Формальный параметр $p$ консервативно берется из разделяемой памяти
    \[ p \leftarrow \SHARED(\M_0). \]

    Чтение обычного поля $\mathrm{f}$ из переменной $a$ использует
    \M-переменную:
    \[ x \leftarrow \GETFIELD(\M_i, a, \mathrm{f}), \]
    аналогично имеем чтение статического поля $\mathrm{T.f}$:
    \[ x \leftarrow \GETSTATIC(\M_i, \mathrm{T.f}). \]
    Запись переменной $x$ в обычное поле $\mathrm{f}$ переменной $a$ порождает
    \M-переменную:
    \[ \M_j \leftarrow \PUTFIELD(\M_i, a, \mathrm{f}, x), \]
    аналогично порождает \M-переменную запись в статическое поле $\mathrm{T.f}$:
    \[ \M_j \leftarrow \PUTSTATIC(\M_i, \mathrm{T.f}, x). \]

    Чтение и запись $i$-го элемента массива консервативно преобразуется в
    работу с синтетическим полем $\mathrm{elements}$, которое добавляется всем
    типам-массивам.

    Вызов функции с параметрами $p_0, \ldots, p_n$ и возвращаемым значением,
    которое записывается в переменную $x$, консервативно представляется
    следующим образом:
    \begin{align*}
      \M_j &\leftarrow \ESCAPE(\M_i, p_0, \ldots, p_n), \\
      \M_k &\leftarrow \RELOAD(\M_j), \\
      x  &\leftarrow \SHARED(\M_k).
    \end{align*}
    Заметим, что в случае отсутствия возращаемого значения, последнюю строку
    необходимо убрать.

    В соответствии с моделью памяти языка \java, необходимо перечитывать
    значения полей объектов перед чтением \eng{volatile} полей и при входе в
    блок синхронизации:
    \[ \M_j \leftarrow \RELOAD(\M_i). \]


  \newpage
  \bibliography{../../biblio}

\end{document}

