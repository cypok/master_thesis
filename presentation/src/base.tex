\documentclass[12pt]{article}

%%%%%%%%%%%%%%%%%%%%%%%%%%%%%%%
% PACKAGES
%%%%%%%%%%%%%%%%%%%%%%%%%%%%%%%

\usepackage[pdftex,unicode,hidelinks,final]{hyperref}
\usepackage{ucs}
\usepackage[utf8x]{inputenc}
\usepackage[english,russian]{babel}
\usepackage{indentfirst}

\usepackage{geometry}

\usepackage[usenames,dvipsnames]{xcolor}

\usepackage{xspace} % for \xspace

\usepackage{totcount} % for \newtotcounter & \total

\usepackage{amssymb} % for mathbb

\usepackage{../../common/cypokcommon}

%%%%%%%%%%%%%%%%%%%%%%%%%%%%%%%
% DOCUMENT-SPECIFIC COMMANDS
%%%%%%%%%%%%%%%%%%%%%%%%%%%%%%%

% Provide some basic information about current Git revision.
%
% TODO: merge with gitinfo package?
% http://www.ctan.org/tex-archive/macros/latex/contrib/gitinfo

\InputIfFileExists{../../common/gitinfo-data}{}{
  % defaults
  \newcommand{\GitMissing}{(no Git info)}
  \newcommand{\GitAbbrHash}{\GitMissing}
  \newcommand{\GitDate}{\GitMissing}
  \newcommand{\GitSubject}{\GitMissing}
}



\newcommand{\java}{\eng{Java}\xspace}

\newcommand{\M}{\ensuremath{\mathbb{M}}}

\newtotcounter{slidescount}
\newcommand{\slide}[1]{%
  \stepcounter{slidescount}%
  \textcolor{Blue}{[Слайд \arabic{slidescount}/\total{slidescount}: #1]}}

%%%%%%%%%%%%%%%%%%%%%%%%%%%%%%%
% TEXT FORMAT
%%%%%%%%%%%%%%%%%%%%%%%%%%%%%%%

\geometry{a4paper,top=2cm,bottom=2cm,left=2.5cm,right=2.5cm,bindingoffset=0cm}
\linespread{1.3}

%%%%%%%%%%%%%%%%%%%%%%%%%%%%%%%
% BODY
%%%%%%%%%%%%%%%%%%%%%%%%%%%%%%%

\title{
  Презентация работы на тему <<Применение анализа указателей и синонимов
  для оптимизации многопоточных программ>>
}
\author{
  Владимир Парфиненко
}

\begin{document}
  {\Large Текст презентации к защите, 5 минут.}

  Git date: \GitDate.
  \medskip\hrule

  \paragraph{Кто я.}
  \slide{титульный с названием и фамилиями моей и научрука}
  Здравствуйте, меня зовут Владимир Парфиненко и тема моей работы
  <<Применение анализа указателей и синонимов для оптимизации многопоточных
  программ>>.

  \paragraph{Что такое анализ синонимов и указателей.}
  \slide{анализ синонимов и... указателей (с английскими названиями)}
  В моей работе рассматривается анализ синонимов, который предоставляет
  информацию о том, могут ли два выражения ссылочного типа быть синонимами, то
  есть ссылаться на один и тот же объект. Анализ синонимов тесно связан
  с анализом указателей, который предоставляет более общую информацию:
  множество всех объектов, на которые может ссылаться выражение.

  \paragraph{Постановка задачи.}
  \slide{цель и задачи}
  Теперь, зная основную терминологию, мы можем сформулировать цель данной
  работы: разработка алгоритма анализа указателей и синонимов для языка \java,
  учитывающего особенности языка и его модели памяти. Для этого необходимо
  изучить существующие алгоритмы анализа, сравнить их точность, эффективность и
  разработать усовершенствованный алгоритм, выбрав за основу один из
  существующих. Особое внимание требуется уделить способу выражения неявных
  зависимостей по данным между операциями работающими с памятью, очень важно
  согласовать этот способ с моделью памяти языка \java.

  \paragraph{Предыдущая работа.}
  В рамках работы на соискание степени бакалавра физики, выполняемой на кафедре
  АФТИ, мною уже был разработан алгоритм анализа указателей для языка \java. Он
  основывается на алгоритме Андерсена, является внутрипроцедурным,
  нечувствительным к потоку управления и использует типовый анализ.
  \slide{пример с отсутствием чувствительности к потоку}
  Как выяснялось позже, алгоритм при анализе многих программ ведет себя через
  чур консервативно, не смотря на предпринятые улучшения. Это было связано
  во-первых с отсутсвием чувствительности к потоку управления. В данном примере
  легко увидеть, что $a$ и $b$ ссылаются на заведомо разные объекты и не могут
  быть синонимами. Однако разработанный алгоритм получал консервативный
  результат, что $a$ и $b$ синонимы, так как их значения получаются из одного и
  того же поля $f$ переменной $x$.

  \paragraph{Проблемы нучувствительного к потоку алгоритма.}
  Так как проведение чувствительного к потоку управления анализа до сих пор
  является неприемлимым в промышленном компиляторе из-за временной/емкостной
  сложности, необходимо было придумать другой способ решения этой проблемы.
  \slide{сравнение видения программы чувствительным и нечувствительным к потоку
  анализом}
  Довольно быстро стало понятно, что проблема не столько в самом алгоритме
  анализа, сколько в нехватке информации во внутреннем представлении, так как
  нечуствительный к потоку управления анализ работает с программой как с
  неупорядоченным множеством операций, и две операции чтения поля $f$
  переменной $x$ для такого анализа абсолютно идентичны и никак не зависят от
  операции записи в поле.

  \paragraph{\M"=переменная.}
  \slide{пример программы с \M"=переменной}
  Описанную проблему удалось решить с помощью введения в программу специальной
  переменной, \M"=переменной \engdef{memory}, олицетворяющей образ всей памяти.
  Таким образом все операции читающие память должны иметь своим аргументом
  \M"=переменную, а операции пишущие в память должны возвращать \M"=переменную.
  На ряду с обычными переменными \M"=переменная версионируется при переводе в
  SSA-форму, что и дает необходимые явные зависимости между операциями
  чтения/записи полей: видно, что два чтения поля в строке 1 и 8 не идентичны,
  они работают с разными версиями \M"=переменной, и алгоритм анализа может
  получить более точные результаты о синонимичности переменных $u$ и $v$.

  \paragraph{Операции языка \java}
  \slide{примитивные операции}
  С помощью небольшого набора примитивных операций, работающих с
  \M"=переменной, можно выразить все операции языка \java, работающие с
  переменными ссылочного типа и памятью, при этом сохранив всю информацию,
  необходимую для проведения анализа указателей. Заметим, что эти операции
  также выражают все свойства модели памяти языка \java.

  \paragraph{Алгоритм анализа.}
  Теперь можно кратко описать работу алгоритма анализа указателей.
  \slide{алгоритм анализа}
  Для анализа конкретного метода первым делом строится вспомогательное
  внутреннее представление, содержащее операции работающие с \M"=переменной.
  Затем алгоритм проводит потоковый анализ над этим операциями: имея изначально
  пустые множества целей переменных ссылочного типа, происходят итерации по
  множеству операций, во время которых каждая операция интерпретируется в
  соответствии с ее семантикой, подробно описанной в тексте работы. Итерации
  проводятся пока не будет достигнута неподвижная точка. Доказательство
  сходимости и корректности данного алгоритма анализа в данной работе не
  приводится.

  \paragraph{Заключение.}
  \slide{что сделано}
  В ходе работы был проведен анализ существующих алгоритмов анализа указателей,
  разработана схема явного выражения зависимостей по памяти. Эта схема была
  применена для повышения точности алгоритма анализа, что также позволило
  точнее отобразить семантику модели памяти языка \java. Прототип алгоритма,
  способный корректно анализировать методы без циклов по \M"=переменным, был
  реализован в рамках проекта
  \eng{Excelsior~RVM}.

  \paragraph{Дальнейшая работа.}
  \slide{что планируется сделать}
  В дальнейшем, в рамках магистерской работы планируется доказать сходимость и
  корректность алгоритма и дать оценки его временной и емкостной сложности.
  Также требуется проработать внутреннее представление для хранения результатов
  анализа, что особенно важно в связи с введением версий \M"=переменной.
  Наконец, планируется реализовать полнофункциональную версию алгоритма в
  рамках проекта \eng{Excelsior~RVM}.

  \paragraph{Конец.}
  \slide{спасибо за внимание}
  Спасибо за внимание, я готов ответить на ваши вопросы.

\end{document}
