\documentclass[xcolor=svgnames,usepdftitle=false]{beamer}

%%%%%%%%%%%%%%%%%%%%%%%%%%%%%%%
% PACKAGES
%%%%%%%%%%%%%%%%%%%%%%%%%%%%%%%

\usepackage[T2A]{fontenc}
\usepackage[utf8]{inputenc}
\usepackage[english,russian]{babel}

\usepackage[babel,final,protrusion=true,expansion]{microtype}

\usepackage{ifdraft}

\usepackage{xspace}

\usepackage{algpseudocode}
\newcommand{\algorithmictitle}[1]{\hspace{8mm}\textbf{#1}}
\algrenewcommand{\algorithmiccomment}[1]{\hfill #1}

\usepackage{ulem}
\let\oldsout\sout
\renewcommand{\sout}[1]{\text{\oldsout{\ensuremath{#1}}}}
\renewcommand<>{\sout}[1]{\alt#2{\beameroriginal{\sout}{#1}}{#1}}

\hypersetup{unicode}

\usepackage{tikz}
\usetikzlibrary{shapes}

\usepackage{../../common/cypokcommon}

%%%%%%%%%%%%%%%%%%%%%%%%%%%%%%%
% DOCUMENT-SPECIFIC COMMANDS
%%%%%%%%%%%%%%%%%%%%%%%%%%%%%%%

% Provide some basic information about current Git revision.
%
% TODO: merge with gitinfo package?
% http://www.ctan.org/tex-archive/macros/latex/contrib/gitinfo

\InputIfFileExists{../../common/gitinfo-data}{}{
  % defaults
  \newcommand{\GitMissing}{(no Git info)}
  \newcommand{\GitAbbrHash}{\GitMissing}
  \newcommand{\GitDate}{\GitMissing}
  \newcommand{\GitSubject}{\GitMissing}
}



\newenvironment{backup}
  {
    \appendix
    \newcounter{framenumberappendix}
    \setcounter{framenumberappendix}{\value{framenumber}}
  }
  {
    \addtocounter{framenumberappendix}{-\value{framenumber}}
    \addtocounter{framenumber}{\value{framenumberappendix}}
  }

\newcommand{\java}{\eng{Java}\xspace}

\newcommand{\M}{\ensuremath{\mathbf{M}}}
\newcommand{\Ms}{\ensuremath{\widebar{\mathbf{M}}}}
\newcommand{\Mf}[1]{\ensuremath{\mathbf{#1}}}

\newcommand{\AOTyped}[1]{O^{\type{#1}}}
\newcommand{\AO}{O}
\newcommand{\AOGlobal}{\AO_{global}}

\newcommand{\type}[1]{\mathsf{#1}}
\newcommand{\field}[1]{\mathsf{#1}}
\newcommand{\sfield}[2]{\type{#1}.\field{#2}}
\newcommand{\method}[1]{\mathsf{#1}}

\newcommand{\op}[1]{\mathbf{#1}}
\newcommand{\pts}[1]{\widebar{#1}}

%%%%%%%%%%%%%%%%%%%%%%%%%%%%%%%
% SLIDES FORMAT
%%%%%%%%%%%%%%%%%%%%%%%%%%%%%%%

% custom footer: author | title | frame number
% same as infolines outer theme but frame number is minimal
\makeatletter
\setbeamertemplate{footline}
{%
  \leavevmode%
  \hbox{%
  \begin{beamercolorbox}[wd=.46\paperwidth,ht=2.25ex,dp=1ex,center]{author in head/foot}%
    \usebeamerfont{author in head/foot}\insertshortauthor
  \end{beamercolorbox}%
  \begin{beamercolorbox}[wd=.46\paperwidth,ht=2.25ex,dp=1ex,center]{title in head/foot}%
    \usebeamerfont{title in head/foot}\insertshorttitle
  \end{beamercolorbox}%
  \begin{beamercolorbox}[wd=.08\paperwidth,ht=2.25ex,dp=1ex,leftskip=0cm plus1fill,rightskip=.015\paperwidth]{page number in head/foot}%
    \usebeamerfont{page number in head/foot}
    \insertframenumber{}\,/\,\inserttotalframenumber
  \end{beamercolorbox}}%
  \vskip0pt%
}
\makeatother

% hide navigation symbols
\beamertemplatenavigationsymbolsempty

\setbeamercolor{structure}{fg=DarkBlue}
\usecolortheme{seahorse}

% custom title page
\setbeamertemplate{title page}[default][rounded=true]
\setbeamercolor*{title}{parent=palette primary}
\setbeamerfont{institute}{size={}}

% custom footer colors: renumber palette uses
\setbeamercolor*{author in head/foot}{parent=palette primary}
\setbeamercolor*{title in head/foot}{parent=palette secondary}
\setbeamercolor*{page number in head/foot}{parent=palette tertiary}

%\setbeamercolor*{block title}{bg=black!10}
%\setbeamercolor*{block body}{bg=black!10}

\setbeamercovered{transparent}

%%%%%%%%%%%%%%%%%%%%%%%%%%%%%%%
% BODY
%%%%%%%%%%%%%%%%%%%%%%%%%%%%%%%

% set PDF info
\titleauthor{%
  Применение анализа указателей и синонимов
  для оптимизации многопоточных программ%
}{%
  Владимир Парфиненко%
}

% set footer info
\renewcommand\insertshortauthor{Парфиненко Владимир Владимирович}
\renewcommand\insertshorttitle{Анализ указателей и синонимов}

% set title page info
\renewcommand\insertauthor{%
  Парфиненко Владимир Владимирович \\
  \vspace{0.5cm}
  \small
    Научные руководители:\\
    м.\,н.\,с.~ИСИ~СО~РАН, Павлов~Павел~Евгеньевич \\
    зав.\,лаб.~ИСИ~СО~РАН, к.\,т.\,н.~Шелехов~Владимир~Иванович%
}
\renewcommand\inserttitle{%
  Применение анализа указателей и синонимов
  для оптимизации многопоточных программ%
}
\institute{%
  Новосибирский национальный исследовательский\\
  государственный университет%
}
\date{
  Новосибирск, 2013\ifdraft{ \\\medskip \footnotesize Git date: \GitDate.}{}
}

\begin{document}

\begin{frame}[plain]
  \titlepage
\end{frame}

\begin{frame}{Анализ синонимов и указателей}

  \begin{block}{Анализ синонимов \engdef{alias analysis}}
    Могут ли два разных выражения ссылочного типа указывать на одно и то же
    место в памяти?
  \end{block}

  \begin{block}{Анализ указателей \engdef{points-to analysis}}
    На какие объекты в памяти могут указывать выражения ссылочного типа?
  \end{block}

\end{frame}

\begin{frame}{Цель и задачи}

  \begin{block}{Цель}
    Разработать алгоритм анализа указателей для языка \java.
  \end{block}

  \begin{block}{Основные задачи}
    \begin{itemize}
      \item Изучить существующие алгоритмы и разработать новый, выбрав
            подходящий за основу.
      \item Разработать схему выражения неявных зависимостей по памяти.
      \item Провести замеры эффективности оптимизации, использующей
            результаты анализа.
    \end{itemize}
  \end{block}

\end{frame}

\begin{frame}{Недостатки предыдущей работы}

  \begin{center}
    \begin{minipage}[t]{0.35\textwidth}
      \begin{block}{Проблема с полями}
        \begin{algorithmic}[1]
          \State $x \gets \op{new}~\type{T}$
          \State $x.f \gets \op{new}~\type{Object}$
          \State $a \gets x.f$
          \State $x.f \gets \op{new}~\type{Object}$
          \State $b \gets x.f$
        \end{algorithmic}
      \end{block}
    \end{minipage}

    \bigskip
    $a$ и $b$~--- могут ли быть синонимами?

    \visible<2->{Да.}
  \end{center}

\end{frame}

\begin{frame}{Суть нечувствительного к потоку управления анализа}

  \centerline{В\'{и}дение программы анализом, который к потоку управления \ldots}

  \begin{columns}[t]
    \begin{column}{0.5\textwidth}
      \begin{block}{\ldots чувствителен}
        \begin{algorithmic}[1]
          \State $x \gets \op{new}~\type{T}$
          \State $x.f \gets \op{new}~\type{Object}$
          \State $a \gets x.f$
          \State $x.f \gets \op{new}~\type{Object}$
          \State $b \gets x.f$
        \end{algorithmic}
      \end{block}
    \end{column}
    \begin{column}{0.5\textwidth}
      \begin{block}{\ldots нечувствителен}
        \begin{algorithmic}
          \State $x \gets \op{new}~\type{T}$
          \State $x.f \gets \op{new}~\type{Object}$
          \State $x.f \gets \op{new}~\type{Object}$
          \State $a \gets x.f$
          \State $b \gets x.f$
        \end{algorithmic}
      \end{block}
    \end{column}
  \end{columns}

\end{frame}

\begin{frame}{\M"=переменная}

  \begin{columns}[t]
    \begin{column}{0.4\textwidth}
      \begin{block}{Исходная программа}
        \begin{algorithmic}[1]
          \State $u \gets a.\field{f}$
          \If{(\ldots)}
            \State $a.\field{f} \gets x$
          \Else
            \State $a.\field{f} \gets y$
          \EndIf
          \State
          \State $v \gets a.\field{f}$
        \end{algorithmic}
      \end{block}
    \end{column}
    \begin{column}{0.6\textwidth}
      \begin{block}{Программа с \M"=переменной}
        \begin{algorithmic}
          \State $u \gets \op{getfield}[\field{f}](\M_0, a)$
          \If{(\ldots)}
            \State $\M_1 \gets \op{putfield}[\field{f}](\M_0, a, x)$
          \Else
            \State $\M_2 \gets \op{putfield}[\field{f}](\M_0, a, y)$
          \EndIf
          \State $\M_3 \gets \op{phi}(\M_1, \M_2)$
          \State $v \gets \op{getfield}[\field{f}](\M_3, a)$
        \end{algorithmic}
      \end{block}
    \end{column}
  \end{columns}

\end{frame}

\begin{frame}{Примитивные операции}

  \begin{columns}[t]
    \begin{column}{0.5\textwidth}
      \begin{align*}
        &x \gets \op{null } \\
        &x \gets \op{getfield}[\field{f}](\M_0, a) \\
        &x \gets \op{new}[\type{T}] \\
        &x \gets \op{shared}[\type{T}](\M_0) \\
        &x \gets \op{cast}[\type{T}](a) \\
        &x \gets \op{phi}(a_1, \ldots, a_N) \\
      \end{align*}
    \end{column}
    \begin{column}{0.5\textwidth}
      \begin{align*}
        &\M_0 \gets \op{initialmemory } \\
        &\M_1 \gets \op{putfield}[\field{f}](\M_0, a, x) \\
        &\M_1 \gets \op{reload}(\M_0) \\
        &\M_1 \gets \op{escape}(\M_0, a) \\
        &\M_x \gets \op{phi}(\M_1, \ldots, \M_N) \\
      \end{align*}
    \end{column}
  \end{columns}

\end{frame}

\begin{frame}{Граф зависимостей по данным}

  \begin{columns}[c]
    \begin{column}{0.67\textwidth}
      \begin{figure}%
        \tikzstyle{op}=[ellipse,draw,line width=0.5pt,font=\tiny]

        \tikzstyle{idx} = [midway,font=\tiny]

        \tikzstyle{value} = [-latex,line width=0.5pt,red]
        \tikzstyle{memory} = [-latex,line width=0.5pt,blue]

        \begin{tikzpicture}
          \foreach \name/\pos/\val in {%
            {inmem/(0,1.5)/$\op{initialmemory}$},
            {pf1/(-1,0)/$\op{putfield}[\field{f}]$},
            {pf2/(1,0)/$\op{putfield}[\field{f}]$},
            {phi/(0,-1.5)/$\op{phi}$},
            {gf/(0,-2.5)/$\op{getfield}[\field{f}]$},
            {a/(-2.5,2)/$\op{new}[\type{A}]$},
            {b/(2,1)/$\op{new}[\type{B}]$},
            {arg/(-2,1)/$\op{shared}[\type{C}]$}%
          }{
            \node[op] (\name) at \pos {\val};
          }

          \foreach \src/\dst/\idx/\kind/\pos in {%
            inmem/pf1/1/memory/right,
            inmem/pf2/1/memory/left,
            pf1/phi/1/memory/right,
            pf2/phi/2/memory/left,
            phi/gf/1/memory/right,
            b/pf2/3/value/right,
            arg/pf1/3/value/left%
          }{
            \draw[\kind] (\src) -- (\dst) node[idx,\pos] {\idx};
          }

          \draw[memory] (inmem) .. controls +(90:0.5cm) and +(90:1cm) .. (arg)
            node[idx,above] {1};

          \draw[value] (a) .. controls +(-15:1cm) and +(90:1.5cm) .. (pf1)
            node[idx,left] {2};
          \draw[value] (a) .. controls +(10:1cm) and +(70:3cm).. (pf2)
            node[idx,above] {2};
          \draw[value] (a) .. controls +(-135:2cm) and +(160:1cm) .. (gf)
            node[idx,below] {2};

        \end{tikzpicture}
      \end{figure}
    \end{column}
    \begin{column}{0.33\textwidth}
      \begin{block}{Программа}
        \begin{algorithmic}[1]
          \State $a \gets \op{new}~\type{A}$
          \If{(\ldots)}
            \State $a.\field{f} \gets \op{arg}_{1,\type{C}}$
          \Else
            \State $a.\field{f} \gets \op{new}~\type{B}$
          \EndIf
          \State $b \gets a.\field{f}$
        \end{algorithmic}
      \end{block}
    \end{column}
  \end{columns}

  \medskip
  \begin{center}
    \small
    Красные стрелки~--- зависимости по значению. \\
    Синие стрелки~--- зависимости по памяти.
  \end{center}

\end{frame}

\begin{frame}{Абстрактные объекты}

  \begin{center}
    \begin{minipage}[t]{0.70\textwidth}
      \begin{block}{Разные источники объектов}
        \begin{algorithmic}[1]
          \State $x \gets \op{new}[\type{T}]$
            \Comment{$\pts{x} = \{\AOTyped{T}_1\}$}
          \State $y \gets \op{new}[\type{T}]$
            \Comment{$\pts{y} = \{\AOTyped{T}_2\}$}
          \State $u \gets \op{shared}[\type{T}](\M)$
            \Comment{$\pts{u} = \{\AOGlobal\}$}
          \State $v \gets \op{shared}[\type{T}](\M)$
            \Comment{$\pts{v} = \{\AOGlobal\}$}
        \end{algorithmic}
      \end{block}

    \end{minipage}
  \end{center}

  \medskip
  \centerline{$AbsObjects = \{\AOTyped{T}_1, \AOTyped{T}_2, \AOGlobal\}$}

\end{frame}

\begin{frame}{Потоковые свойства}

  \[\pts{x} \subseteq AbsObjects\]

  \pause

  \begin{align*}
    & \Ms = (\\
    & \quad \Mf{fields} \subseteq
         AbsObjects \times Fields \times \powerset{AbsObjects}, \\
    & \quad \Mf{shared} \subseteq AbsObjects \\
    & )
  \end{align*}

\end{frame}

\begin{frame}{Пример потоковых функций}

    \begin{gather*}
      \M' \gets \op{escape}(\M, a) \colon \\
      \begin{aligned}
        &\Ms'.\Mf{fields} = \Ms.\Mf{fields}, \\
        &\Ms'.\Mf{shared} = \Ms.\Mf{shared} \cup \mathbf{A}^+_{\Ms}(\pts{a}).
      \end{aligned}
    \end{gather*}

  \begin{gather*}
    x \gets \op{getfield}[\field{f}](\M, a) \colon \\
    \pts{x} = \left( \bigcup_{\AO \in \pts{a}} \Ms.\Mf{fields}(\AO,
    \field{f}) \right) \cup
    \begin{cases}
      \Ms.\Mf{shared}, & \text{если } \pts{a} \cap
        \Ms.\Mf{shared} \ne \emptyset; \\
      \emptyset, & \text{иначе}.
    \end{cases}
  \end{gather*}

\end{frame}

\begin{frame}{Результаты анализа}

  Выражения ссылочного типа могут быть синонимами тогда и только тогда, когда
  их множества целей, вычисленные с помощью анализа потока данных, имеют
  непустое пересечение.

\end{frame}

\begin{frame}{Оптимизация удаления избыточных чтений}
  \begin{center}
    \begin{minipage}[t]{0.45\textwidth}
      \begin{block}{Избыточное чтение поля}
        \begin{algorithmic}
          \State $a.\field{f} \gets 42$
          \State $b.\field{f} \gets 37$
          \State $x \gets \sout<2->{a.\field{f}}
                          \only<2->{\sout<3->{\ 42}}
                          \only<3->{\ ???}$
        \end{algorithmic}
      \end{block}
    \end{minipage}

    \bigskip

    \visible<2->{Если $a$ и $b$ не могут быть синонимами.}

    \visible<3->{А если $a$ указывает на разделяемую память?}
  \end{center}
\end{frame}

\begin{frame}{Модель памяти языка Java}
  \begin{columns}[t]
    \begin{column}{.5\textwidth}
      \begin{block}{Чтение из \eng{volatile} поля}
        \begin{algorithmic}
          \State $a.\field{x} \gets 42$
          \State $\ldots$
          \State $t \gets b.\field{volatileField}$
          \State $\ldots$
          \State $x \gets a.\field{x}$
        \end{algorithmic}
      \end{block}
    \end{column}
    \begin{column}{.5\textwidth}
      \begin{block}{Чтение из обычного поля}
        \begin{algorithmic}
          \State $a.\field{x} \gets 42$
          \State $\ldots$
          \State $t \gets b.\field{normalField}$
          \State $\ldots$
          \State $x \gets \sout<2->{a.\field{x}}
                          \only<2->{\ 42}$
        \end{algorithmic}
      \end{block}
    \end{column}
  \end{columns}
\end{frame}

\begin{frame}{Практические результаты}

  \begin{itemize}
    \item Алгоритм анализа реализован в статическом компиляторе \java байткода
          в рамках проекта \eng{Excelsior Research Virtual Machine}.
    \item Реализована оптимизация удаления избыточных чтений полей объектов,
          статических полей и элементов массивов.
  \end{itemize}
\end{frame}

\begin{frame}{Практические результаты}

  \begin{itemize}
    \item Замерены результаты оптимизации при использовании консервативного
          алгоритма анализа указателей (\eng{C}) и разработанного алгоритма
          (\eng{D}).
  \end{itemize}

  \begin{table}
    \begin{tabular}{|l|c|c|c|c|}\hline
      {\small Количество удаленных чтений} \ldots & C & D & Прирост \\ \hline
      \ldots полей объектов
      & \num{5366} & \num{7546} & $+41\%$ \\ \hline
      \ldots статических полей
      & \num{472}  & \num{472}  & $0\%$ \\ \hline
      \ldots элементов массивов
      & \num{315}  & \num{659}  & $+109\%$ \\ \hline
    \end{tabular}
    \caption{Результаты для набора \eng{SPECjvm2008}}
  \end{table}

\end{frame}

\begin{frame}{Результаты}

  Сделано следующее:
  \begin{itemize}
    \item изучены существующие алгоритмы анализа указателей;
    \item разработана схема выражения неявных зависимостей по памяти с помощью
          \M"=переменной;
    \item разработан алгоритм, проводящий нечувствительный к потоку анализ на
          графе зависимостей по данным;
    \item разработана оптимизация удаления избыточных чтений, учитывающая
          особенности модели памяти языка \java;
    \item реализованы алгоритм анализа и алгоритм оптимизации в рамках проекта
          \eng{Excelsior Research Virtual Machine}.
  \end{itemize}

  Получен диплом 1 степени на 51-ой международной научной студенческой
  конференции <<Студент и научно-технический прогресс>>.

\end{frame}

\begin{frame}{Дальнейшая работа}

  Планируется сделать следующее:
  \begin{itemize}
    \item внедрить в промышленный компилятор \eng{Excelsior JET},
    \item адаптировать для межпроцедурного анализа.
  \end{itemize}

\end{frame}

\begin{frame}{Конец}
  \centerline{\huge Спасибо за внимание!}
\end{frame}

\begin{backup}

\begin{frame}{Запасные слайды}
  \centerline{\ldots}
\end{frame}

\begin{frame}{Представление потоковых свойств}

  \begin{block}{Внутреннее представление}
    \begin{algorithmic}
      \State $\M_0 \gets \op{initialmemory}$
        \visible<3->{\Comment{$\M_0 = \{\Mf{fields}\colon}
          \{ \AO_1.\field{f} \colon \emptyset \}\}$}
      \State $a \gets \op{new}[\type{A}]$
        \visible<2->{\Comment{$\pts{a} = \{\AO_1\}$}}
      \State
        \State $x_1 \gets \op{new}[\type{A}]$
          \visible<2->{\Comment{$\pts{x_1} = \{\AO_2\}$}}
        \State $\M_1 \gets \op{putfield}[\field{f}](\M_0, a, x_1)$
          \visible<3->{\Comment{$\M_1 = \{\Mf{fields}\colon}
            \{ \AO_1.\field{f} \colon \{ \AO_2 \} \}\}$}
      \State
        \State $x_2 \gets \op{shared}(\M_0)$
          \visible<2->{\Comment{$\pts{x_2} = \{\AOGlobal\}$}}
        \State $\M_2 \gets \op{putfield}[\field{f}](\M_0, a, x_2)$
          \visible<3->{\Comment{$\M_2 = \{\Mf{fields}\colon}
            \{ \AO_1.\field{f} \colon \{ \AOGlobal \} \}\}$}
      \State
      \State $\M_3 \gets \op{phi}(\M_1, \M_2)$
        \visible<3->{\Comment{$\M_3 = \{\Mf{fields}\colon}
          \{ \AO_1.\field{f} \colon \{ \AO_2, \AOGlobal \} \}\}$}
      \State $y \gets \op{getfield}[\field{f}](\M_3, a)$
        \visible<4->{\Comment{$\pts{y} = \{ \AO_2, \AOGlobal \}$}}
    \end{algorithmic}
  \end{block}

\end{frame}

\begin{frame}{Проблема многопоточности}

  \begin{columns}[t]
    \begin{column}{.5\textwidth}
      \begin{block}{Однопоточная программа}
        \begin{algorithmic}[1]
          \State $b \gets \op{new}~\type{A}$
          \State $b.\field{f} \gets 37$
          \State $\ldots$
          \State $\op{escape}(b)$
          \State $\ldots$
          \State $x \gets \sout<2->{b.\field{f}}
                          \only<2->{\ 37}$
          \State $\ldots$
          \State $y \gets \sout<2->{b.\field{f}}
                          \only<2->{\ 37}$
        \end{algorithmic}
      \end{block}
    \end{column}
    \begin{column}{.5\textwidth}
      \begin{block}{Многопоточная программа}
        \begin{algorithmic}[1]
          \State $b \gets \op{new}~\type{A}$
          \State $b.\field{f} \gets 37$
          \State $\ldots$
          \State $\op{escape}(b)$
          \State $\only<-3>{\ldots}
                  \only<4->{\op{while}~(\ldots)\ \ldots}$
          \State $x \gets \sout<3->{b.\field{f}}
                          \only<3->{\sout<4->{\ 37}}
                          \only<4->{\ b.\field{f}}$
          \State $\only<-4>{\ldots}
                  \only<5->{\op{while}~(\ldots)\ \ldots}$
          \State $y \gets \sout<3->{b.\field{f}}
                          \only<3->{\sout<4->{\ 37}}
                          \only<4->{\sout<5->{\ x}}
                          \only<5->{\ b.\field{f}}$
        \end{algorithmic}
      \end{block}
    \end{column}
  \end{columns}

\end{frame}

\begin{frame}{Проблема многопоточности}

  \begin{center}
    \begin{minipage}[t]{0.60\textwidth}
      \begin{block}{Многопоточная программа + \eng{JMM}}
        \begin{algorithmic}[1]
          \State $b \gets \op{new}~\type{A}$
          \State $b.\field{f} \gets 37$
          \State $\ldots$
          \State $\op{escape}(b)$
          \State $\only<-2>{\ldots}
                  \only<3>{\op{while}~(\ldots)\ \ldots}
                  \only<4->{\op{while}~(!\sfield{Data}{ready})\ \ldots}$
          \State $x \gets \sout<2->{b.\field{f}}
                          \only<2->{\sout<4->{\ 37}}
                          \only<4->{\ b.\field{f}}$
          \State $\only<-2>{\ldots}
                  \only<3->{\op{while}~(\ldots)\ \ldots}$
          \State $y \gets \sout<2->{b.\field{f}}
                          \only<2->{\sout<4->{\ 37}}
                          \only<4->{\ x}$
        \end{algorithmic}
      \end{block}
      \onslide+<4->{
        $\sfield{Data}{ready}$~--- статическое \eng{volatile} поле.}
    \end{minipage}
  \end{center}

\end{frame}

\begin{frame}{Некорректно синхронизованные программы}

  $\sfield{Shared}{data}$, $\sfield{Shared}{ready}$~--- обычные статические
  поля.

  \begin{columns}[t]
    \begin{column}{.5\textwidth}
      \begin{block}{Поток 1 <<Производитель>>}
        \begin{algorithmic}[1]
          \State $\sfield{Shared}{data} \gets \op{calc}(\ldots)$
          \State $\sfield{Shared}{ready} \gets \op{true}$
        \end{algorithmic}
      \end{block}
    \end{column}
    \begin{column}{.5\textwidth}
      \begin{block}{Поток 2 <<Потребитель>>}
        \begin{algorithmic}[1]
          \State $\alt<1>{\vphantom{A}\ldots}{result \gets \sfield{Shared}{data}}$
          \While{$(!\sfield{Shared}{ready})$}
            \State \ldots
          \EndWhile
          \State $\sout<2->{result \gets \sfield{Shared}{data}}$
        \end{algorithmic}
      \end{block}
    \end{column}
  \end{columns}

\end{frame}

\begin{frame}{Внутреннее представление сложных операций}

  \begin{columns}[t]
    \begin{column}{.5\textwidth}
      \begin{align*}
        a \gets \sfield{T}{f} & \\
        (\text{статическое \eng{volatile} поле}) & \\
        & \\
        \op{monitorenter}(\ldots) & \\
        & \\
        c \gets \op{invoke}(\method{foo}, p_1, \ldots, p_N) & \\
        & \\
        & \\
      \end{align*}
    \end{column}
    \begin{column}{.5\textwidth}
      \begin{align*}
        & \M_1 \gets \op{reload}(\M_0) \\
        & a \gets \op{getstatic}(\M_1, \sfield{T}{f}) \\
        & \\
        & \M_2 \gets \op{reload}(\M_1) \\
        & \\
        & \M_3 \gets \op{escape}(\M_2, p_1, \ldots, p_N) \\
        & \M_4 \gets \op{reload}(\M_3) \\
        & c \gets \op{shared}(\M_4) \\
      \end{align*}
    \end{column}
  \end{columns}

\end{frame}

\end{backup}

\end{document}

